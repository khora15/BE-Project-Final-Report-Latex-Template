\chapter{Introduction}

Cloud computing is a model for enabling convenient, on-demand network access to a shared pool of configurable computing resources (e.g., networks, servers, storage, applications, and services) that can be rapidly provisioned and released with minimal management effort or service provider interaction . Cloud computing is a general term for anything that involves delivering hosted services over the Internet. It is the delivery of computing as a service rather than a product, whereby shared resources, software, and information are provided to computers and other devices as a metered service over a network (typically the Internet). 
  Today, Cloud Computing has become an integral part of the I.T industry. Cloud computing is a technology that uses the internet and central remote servers to maintain data and applications. Cloud computing allows consumers and businesses to use applications without installation and access their personal files at any computer with internet access. This technology allows for much more efficient computing by centralizing storage, memory, processing and bandwidth. Cloud Computing is gaining recognition and a large number of users because of its accessibility, reduced complexity and low pay-per-use rates .

\section{Project Idea}


1.  Convenience
Cloud computing allows people to access data and documents from any computer, tablet or smart phone, as long as they have a working internet connection. This is especially helpful when working on collaborative projects, as documents can be simultaneously viewed and edited from widely disparate locations. The automatic software updates that come with cloud software also make it easier to keep up with current regulations and compliance laws.

2. Cost
Cloud computing is more cost-effective than traditional software. Instead of purchasing and installing programs onto various devices, the software exists on a remote server. Use of the software is on a subscription basis rather than purchasing the software outright. In other words, businesses are required to pay for the service only when it is required. This saves money and provides the flexibility to scale up or down as demand fluctuates. For companies that have seasonal spikes, cloud computing provides a distinct advantage over buying software.

3. Storage
Another major benefit from utilizing the cloud is the doing away with server and hard drive constraints. Cloud computing businesses offer varying degrees of storage, and upgrading capacity is simply a matter of paying a higher monthly fee. Conversely, increasing storage space by traditional means would involve purchasing expensive equipment and installation costs. Maintenance expenditures are also eliminated as all the storage equipment is owned by the cloud computing provider. There is no need for a specialist IT staff to fix bugs and install upgrades relative to the software. An example of this type of storage is SugarSync.

4. Security and Backup
Cloud software cannot be pirated as the program is hosted on a single centralized server. These servers are extremely resilient and hosted over multiple countries, making it highly unlikely that data will be lost or inaccessible. Moreover, the cost of security is defrayed as the provider is responsible for maintaining the integrity of the system. As the cloud computing provider’s business is dependent on keeping client data secure, these measures are often complex and impossible to crack.

5. Green Credentials
In a recent study by Microsoft, cloud computing was purported to reduce carbon emissions for businesses by as much as 30 percent. This is primarily due to the energy savings garnered from utilizing the cloud in lieu of an entire in-house server. Going green saves your company money while at the same time can be used as an effective marketing strategy to clients looking for more responsible businesses.


\section{Background}

To use a Cloud, a user has to login into his account at the CSP website through a web browser. For a Cloud user who uses this service regularly and frequently, it is a very tedious job to login through the web browser again and again. In June 2009, a study conducted by VersionOne  found that 41% of senior IT professionals actually don't know what cloud computing is and two-thirds of senior finance professionals are confused by the concept, highlighting the advanced and complicated deployment systems of clouds.
For  most  cloud computing applications, the entire user interface  resides  inside  a  single  window in  a  Web  browser. Several  initiatives aim  to  provide  a  richer  user  experience  for  Internet  applications.  One approach  is  to  exploit  the  cloud computing  paradigm  to  provide  all  the  facilities of an operating system inside a browser.  The eyeOS system, for example. Another  solution  would bypass  the  Web  browser,  substituting a  more-capable  software  system  that runs  as  a  separate  application on  the client computer and communicates directly with servers in the cloud. 


\section{Need Of Project}
Though Cloud computing has progressed very rapidly, it must be accepted that Desktop applications are not going to vanish completely. Hence, there is a need to try to integrate these two models for more efficient utility.
As mentioned earlier, the Cloud user today has his entire dependency on the Web Browser. A frequent user needs to login with his credentials repeatedly for every session. Also, for any novice user, the interface of his Cloud Service Provider maybe difficult to comprehend. 
  The Windows Explorer provides a more familiar working environment to users. The files and folder system has been used for decades. It is easy to visualise these User Interfaces.  Thus, the cloud integrated with Windows explorer provides ease of use and user-friendliness. Moreover, it reduces to need for an intermediary Web browser to access user’s cloud account. 




