\chapter{Research Methodology}

\section{Steps To Acquire and Process}

  Cloud computing is ever- growing. BMC basically works a lot in this field. Cloud computing is the delivery of computing as a service rather than a product, whereby shared resources, software, and information are provided to computers and other devices as a utility (like the electricity grid) over a network (typically the Internet).
Cloud computing entrusts services (typically centralized) with a user's data, software and computation on a published application programming interface (API) over a network. It has considerable overlap with software as a service (SaaS).
End users access cloud based applications through a web browser or a light weight desktop or mobile app while the business software and data are stored on servers at a remote location. Cloud application providers strive to give the same or better service and performance than if the software programs were installed locally on end-user computers.
At the foundation of cloud computing is the broader concept of infrastructure convergence (or Converged Infrastructure) and shared services. This type of data centre environment allows enterprises to get their applications up and running faster, with easier manageability and less maintenance, and enables IT to more rapidly adjust IT resources (such as servers, storage, and networking) to meet fluctuating and unpredictable business demand

\subsection{Infrastructure as a Service (IaaS)}

In this most basic cloud service model, cloud providers offer computers – as physical or more often as virtual machines –, raw (block) storage, firewalls, load balancers, and networks. IaaS providers supply these resources on demand from their large pools installed in data centers. Local area networks including IP addresses are part of the offer. For the wide area connectivity, the Internet can be used or - in carrier clouds - dedicated virtual private networks can be configured.
To deploy their applications, cloud users then install operating system images on the machines as well as their application software. In this model, it is the cloud user who is responsible for patching and maintaining the operating systems and application software. Cloud providers typically bill IaaS services on a utility computing basis, that is, cost will reflect the amount of resources allocated and consumed.

 
figure 4.1  IaaS



\section{Design Pinciples}

\subsection{Application v/s Desktop Integeration}

Cloud Services could be provided to the user via a standalone desktop application or as an integrated  functionality in the windows explorer. The windows explorer provides a more familiar interface of files and folders. Such an integration can be deployed alongwith the OS itself too.  No other  3rd party softwares or platforms are required, if integrated withing Windows explorer itself.
\section{Windows System services}

On Microsoft Windows operating systems, a Windows service is a long-running executable that performs specific functions and which is designed not to require user intervention. Windows services can be configured to start when the operating system is booted and run in the background as long as Windows is running, or they can be started manually when required. They are similar in concept to a Unix daemon. Many appear in the processes list in the Windows Task Manager, most often with a username of SYSTEM, LOCAL SERVICE or NETWORK SERVICE, though not all processes with the SYSTEM username are services. The remaining services run through svchost.exe as DLLs loaded into memory.

\subsection{Developing a Window Service}

A Windows Service is created using devAelopment tools such as Microsoft Visual Studio or Embarcadero Delphi. Windows provides an interface called the Service Control Manager that manages the starting and stopping of services. An application that wants to be a service needs to first be written in such a way that it can handle start, stop, and pause messages from the Service Control Manager. Then, in one or more API calls, the name of the service and other attributes such as its description are registered with the Service Control Manager. Although typically services do not have a user interface, developers can add forms and other UI components. In this case, the "Allow service to interact with desktop" should be checked on the Logon tab in the Service properties dialog (though care should be taken with this approach as this can cause a security risk since any logged in user would be able to interact with the service).

 
System service can be started in 2 ways :

1.  On an event
2.	On booting.

\section{Input}

Amazon EC2 presents a true virtual computing environment, allowing you to use web service interfaces to launch instances with a variety of operating systems, load them with your custom application environment, manage your network’s access permissions, and run your image using as many or few systems as you desire. Also, it provides API s for developers to code for theit own VM’s on Amazon. 
  Though project can work on any OS theoretically, we decided to implement it in Windows XP. Windows is a more user friendly OS and dominant for applications on client side. 




